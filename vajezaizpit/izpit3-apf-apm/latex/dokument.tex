\documentclass[a4paper, 11pt]{article}
\usepackage[utf8]{inputenc}
\usepackage[T1]{fontenc}
\usepackage[slovene]{babel}
\usepackage{lmodern}

\usepackage{amsmath}
\usepackage{amsthm}
\usepackage{amsfonts}

% definicija novega ukaza
\newcommand{\N}{N}

% Peanovi aksiomi
% Mojca Novak

\begin{document}

        % povzetek
        V nadaljevanju je zapisana definicija Peanovih aksiomov.

        % definicija
        Peanovi aksiomi
        Množica naravnih števil je množica $\N$ s funkcijo ??, 
        ki vsakemu naravnemu številu $n$ priredi njegovega neposrednega naslednika $\varphi(n)$. 
        Pri tem veljajo naslednji aksiomi:
        \begin{enumerate}
            \item $\N$ vsebuje število $\epsilon$, ki ni neposredni naslednik nobenega naravnega števila;
            \item neposredna naslednika dveh različnih naravnih števil sta različna, tj.\ funkcija $\varphi$ je injektivna: 
            če je ??, je ??;
            \item Če za podmnožico ?? veljata lastnosti:
            \begin{enumerate}
                \item ?? in
                \item če je ??, je tudi ??,
                potem je $A = \N$.
            \end{enumerate}
        \end{enumerate}
        % konec definicije

        % literatura
    
\end{document}