\documentclass[a4paper, 12pt]{article}
\usepackage[slovene]{babel}
\usepackage{amsfonts,amssymb,amsmath,mathrsfs,amsthm}
\usepackage[utf8]{inputenc}
\usepackage[T1]{fontenc}
\usepackage{url}
\usepackage{hyperref}

\newcommand{\B}{\mathscr{B}^1}
\newcommand{\R}{\mathcal{R}}
\newcommand{\N}{\mathbb{N}}

\theoremstyle{definition}
\newtheorem{zgled}{Zgled}

%%%%%%%%%%%%%%%%%%%%%%%%%%%%%%%%%%%%%%%%%%%%%%%%%%%%%%%%%%%%%%%%%%%%%%%%%%%%%

\def\R{\mathbb{R}}  % mnozica realnih stevil

%%%%%%%%%%%%%%%%%%%%%%%%%%%%%%%%%%%%%%%%%%%%%%%%%%%%%%%%%%%%%%%%%%%%%%%%%%%%%

\title{Thomaeova funkcija}
\author{Beno Učakar}
\date{}

\begin{document}
\maketitle

Oglejmo si Thomaeovo funkcijo, ki je primer funkcije prvega Bairovega razreda.
Take funkcije lahko definiramo na naslednji način.
Naj bo $\emph{D} \subseteq \R$. Funkcija $\emph{f} : \emph{D} \to \R $ je \emph{funkcija prvega Bairovega razreda}, 
če obstaja funkcijsko zaporedje $\{f_n\}$ zveznih funkcij na $D$, ki po točkah konvergira k $f$. 
Ta razred označimo z $\B(D)$ oziroma, če ne bo nevarnosti zmede, kar z $\B$.

\begin{zgled}
Funkcijsko zaporedje $\emph{f}_n :(0,1) \to \R$ definiramo na sledeč način.
    Za vsak $\emph{p},\emph{q} \in \N_0 , 1 \le \emph{q} < \emph{n}$ in $0 \le \emph{p} \le \emph{q} $, $1 \le q < n$ in $0 \le p \le q$ definiramo
    \begin{itemize}
        \item \(f_n(x) = \max\left\{\frac{1}{n}, \frac{1}{q} + 2n^2\left(x - \frac{p}{q}\right)\right\}\) na intervalu \(\left(\frac{p}{q} - \frac{1}{2n^2}, \frac{p}{q}\right)\) in
        \item \(f_n(x) = \max\left\{\frac{1}{n}, \frac{1}{q} - 2n^2\left(x - \frac{p}{q}\right)\right\}\) na intervalu \(\left(\frac{p}{q}, \frac{p}{q} + \frac{1}{2n^2}\right)\).
    \end{itemize}
    V vseh ostalih točkah naj bo $f_n(x) = \frac{1}{n}$.
    Preverimo lahko, da so intervali $\left(\frac{p}{q} - \frac{1}{2n^2}, \frac{p}{q} + \frac{1}{2n^2}\right)$ paroma disjunktni in zgornja definicija je dobra.
    Opazimo, da je $f_n(x)$ odsekoma linearna zvezna. Če vzamemo limito po točkah, dobimo

    $$
    \emph{f}(x) = \begin{cases}
         \frac{1}{q}; \quad x = \frac{p}{q} \: \text{je pokrajšan ulomek za} p, q \in \N \\
        0; \quad \text{\emph{x} je iracionalen}
    \end{cases}
    $$

    Pokazali smo, da \emph{Thomaeova funkcija} pripada $\B$. 
\end{zgled}
    
% konec zgleda

%%%%%%%%%%%%%%%%%%%%%%%%%%%%%%%%%%%%%%%%%%%%%%%%%%%%%%%%%%%%%%%%%%%%%

\bibliographystyle{plain}
\bibliography{viri.bib}
\nocite{*}

\end{document}