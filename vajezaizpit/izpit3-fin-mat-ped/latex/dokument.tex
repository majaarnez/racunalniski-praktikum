\documentclass[a4paper, 11pt]{article}
\usepackage[utf8]{inputenc}
\usepackage[T1]{fontenc}
\usepackage[slovene]{babel}
\usepackage{lmodern}

\usepackage{amsmath}
\usepackage{amsthm}
\usepackage{amsfonts}

\theoremstyle{definition}
\newtheorem{definicija}{Definicija}



\newcommand{\N}{\mathbb{N}}

\title{Peanovi aksiomi}
\author{Mojca Novak}
\date{26. \ 1. \ 2025}

\begin{document}

\maketitle
\begin{abstract}
            V nadaljevanju je zapisana definicija Peanovih aksiomov.

\end{abstract}

        \begin{definicija}
        
            \textbf{Peanovi aksiomi} \cite{zapiski} Množica naravnih števil je množica $N$ s funkcijo $\varphi : \N \to \N$, 
        ki vsakemu naravnemu številu $n$ priredi njegovega neposrednega naslednika $\varphi(n)$. 
        Pri tem veljajo naslednji aksiomi:
        \begin{enumerate}
            \item $N$ vsebuje število $\epsilon$, ki ni neposredni naslednik nobenega naravnega števila;
            \item neposredna naslednika dveh različnih naravnih števil sta različna, tj.\ funkcija $\varphi$ je injektivna: 
            če je $\emph{n} \neq \emph{m}$, je $\varphi(\emph{n}) \neq \varphi(\emph{m})$;
            \item Če za podmnožico $\emph{A} \in \N$ veljata lastnosti:
            \begin{enumerate}
                \item $\epsilon \in \emph{A}$ in
                \item če je $\emph{n} \in \emph{A}$, je tudi $\varphi(\emph{n}) \in \emph{A}$,
                potem je $A = N$.
            \end{enumerate}
        \end{enumerate}
        \end{definicija}
        
        

        \bibliographystyle{plain}
        \bibliography{viri.bib}
    
\end{document}