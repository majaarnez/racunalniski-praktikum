\documentclass[11pt]{article}
\usepackage[a4paper, margin=2.5cm]{geometry}
\usepackage[slovene]{babel}
\usepackage{graphicx}
\usepackage{amsthm}
\usepackage{booktabs}
\usepackage[utf8]{inputenc}
\usepackage[T1]{fontenc}
\usepackage{amssymb}
\usepackage{amsmath}

\newcommand{\f}{\mathcal{F}}

\title{Brownovo gibanje}
\author{Matej Rojec}
\date{}

\begin{document}

\maketitle

\theoremstyle{definition}
\newtheorem{definicija}{Definicija}

\theoremstyle{plain}
\newtheorem{izrek}{Izrek}

Brownovo gibanje (več v \cite{karatzas1991browian}) je intuitivno slučajen proces, % Sklic na knjigo
ki predstavlja naključno gibanje delcev v mediju.

% Slika: PerrinPlot2.pdf
% Napis pod sliko

    
\begin{figure}[ht!]
    \centering
    \includegraphics[width=0.5\textwidth]{PerrinPlot2.pdf}
    \caption{Reprodukcija slike iz Jean Baptiste Perrin, \emph{Mouvement brownien et réalité moléculaire}, Ann. de Chimie et de Physique (VIII) 18, 5-114, 1909}  


    % \label{fig::durer}
  \end{figure}



    \begin{definicija}
        Standardno Brownovo gibanje $\{B_t\}_{t \geq 0}$ je slučajen proces z naslednjimi lastnostmi: 
       \begin{enumerate}
       \item  $B_0 = 0$.
        \item Prirastki $B_{t_n} - B_{t_{n-1}}, B_{t_{n-1}} - B_{t_{n-2}}, \ldots, B_2 - B_1, B_1 - B_0$ so neodvisne slučajne spremenljivke, za vsak $t_0 \leq t_1 \leq \cdots \leq t_{n-1} \leq t_n$.
        \item Za vsak $t \geq 0$ in $h > 0$ velja $B_{t+h} - B_t \sim \mathcal{N}(0, h)$.
        \item Funkcija $t \mapsto B_t$ je zvezna skoraj gotovo.

       \end{enumerate}
       
    \end{definicija}
    
   
    
    Preden zapišemo izrek, definirajmo še pojem časa ustavljanja.
    
    \begin{definicija}
        Slučajna spremenljivka $\tau$ na verjetnostnem prostoru $ (\omega,\f,P)$ z vrednostmi v $ \mathbb{R}^+$
    je čas ustavljanja glede na filtracijo $ (\f_t)_{t \in T}$, če velja $ \forall t \in T : \{\tau \leq t\} \in \f_t$.
    \end{definicija}
    
    Zdaj lahko zapišemo izrek \ref{thm:stopped_brownian}. % Sklic na izrek z oznako 
    
    % Začetek izreka
    \begin{izrek}
    Naj bo $\{B_t\}_{t \geq 0}$ (standardno) Brownovo gibanje, $ \tau$ čas ustavljanja glede na 
    $ (\f_t)_{t \geq 0}$ in naj bo $ P=[\tau < \infty] = 1$.
    Potem je tudi proces:
    \[
    \hat{B} := \{B_{T+t} - B_T \mid t \geq 0\}
    \]
    (standardno) Brownovo gibanje in neodvisen od $ \f_T$.
    \label{thm:stopped_brownian}
    \end{izrek}
    
    % Konec izreka
    

    \bibliographystyle{plain}
    \bibliography{knjiga.bib}

\end{document}